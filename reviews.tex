\documentclass[a4paper,10pt]{scrartcl}

\usepackage{times,graphicx,alltt,enumerate,amsmath,amssymb,eurosym,hyperref,a4wide,cite,xspace,ifthen,url,hyperref,color,soul,pifont}

\newcounter{revnumber}

\newenvironment{review}[1]{
	\addtocounter{revnumber}{1}
	\subsection*{Review \arabic{revnumber} - #1
	\makeatother}
	}      

\newenvironment{myindentpar}[1]%
{\begin{list}{}%
         {\setlength{\leftmargin}{#1}}%
         \item[]%
}
{\end{list}}


\newcommand\serieheader[6]{
	{\noindent \bf \Large \sffamily {#1} {#2}}\\[2mm] {\sf #3} \hfill {\sf #5}\\[2mm] {\sf #4} \hfill {\sf #6} \noindent \rule{\linewidth}{0.3mm}
	}

\newboolean{showcomments}
\setboolean{showcomments}{true}
\ifthenelse{\boolean{showcomments}}
{\newcommand{\nb}[2]{\fbox{\bfseries\sffamily\scriptsize#1}{\sf\small$\blacktriangleright$\textit{#2}$\blacktriangleleft$}}}
{\newcommand{\nb}[2]{}}
\newcommand{\vectornorm}[1]{\left|\left|#1\right|\right|}



\newcommand\CB[1]{\nb{Chris}{#1}}
\newcommand\cb[1]{\nb{Chris}{#1}}
\newcommand\ab[1]{\nb{Alberto}{#1}}
\newcommand\AB[1]{\nb{Alberto}{#1}}




\begin{document}



%%%%%%%%%%%%%%%%%%%%%%%%%%%%%%%%%%%%%%%%%%%%%%%%%%%%%%%%%%%%%%%%%%%%%
\begin{review} {}

\noindent{\bf Summary}

The authors presents the results of an empirical study at Microsoft that
investigated code review practices. The authors use the term ``modern
code review'' meaning that such reviews are performed informally, are
supported with tools, and are performed frequently in everyday software
engineering practice. The authors observed 17 developers in 16 product
teams and analyzed 570 review comments to derive recommendations for
practitioners and researchers working of review processes and tools.
They study the motivations of developers, actual outcomes of reviews,
and challenges experienced. The study shows that besides defect
detection knowledge transfer, team awareness, and improved solutions are
the key factors that motivate developers to perform code reviews.\\

\noindent{\bf Evaluation}

The paper addresses a (still) interesting topic. Structured inspection
techniques proposed in the 1970s are rarely performed in software
development in their original form today. Instead companies are nowadays
using more lightweight techniques to perform code reviews. Analyzing
current review practices and analyzing the benefits and actual use in
current real-world development processes is an important and interesting
topic of research.

The paper is well written and motivated and I liked reading it. I liked
particularly that the authors takes a very practical and realistic view
on code reviews. The research method involving the review of a previous
study, observations, interviews, card sorting techniques, affinity
diagrams and survey for validation is well described and reasonable.

The results rather confirm what can be expected but are not really
surprising in my view. It is known already that code reviewing helps
finding defects, maintaining team awareness, improving code quality, and
transferring knowledge among team members. The recommendations and
implications summarize the detailed results but are also rather
high-level. While interesting I doubt that a lot can be learned.

Regarding threats to validity I agree with the authors that empirical
research done in one company can contribute to scientific development if
it is based on careful observation. However, the study might be biased
by the CodeFlow code review tool that was used by developers participating
in the study. The authors briefly discuss this issue but this discussion
should be extended in my view. Furthermore, only 17 users participated
in the study. The authors write that they sampled across different
product teams to address different perspectives and to cover different
levels of experience. Still 17 participants is not a high number given that
the CodeFlow user base is 28.000 according to the authors.

The authors investigate ad-hoc review processes. However, I missed a
discussion in the paper of how the results relate to more conventional
techniques traditionally used in code reviews. For instance, would
checklists or reading techniques be useful in the described development
context?

The paper lists a number of existing studies and papers on software
inspection and code reviews but also misses important papers from the
literature. The following references should be considered:

\begin{enumerate}

\item Parnas \& Weiss, D.M. Active Design Reviews: Principles and Practice. JSS
7, 259-265 (1987)

\item Perpich et al., Anywhere, Anytime Code Inspections: Using the Web to
Remove Inspection Bottlenecks in Large-Scale Software Development. ICSE
1997

\item Biffl el al., A family of experiments to investigate the effects of
groupware for software inspection. ASE Journal 13(3): 373-394 (2006)

\item Johnson \& Tjahjono, Assessing software review meetings: A controlled
experimental study using CSRS. ICSE 1997

\item Lanubile \& Mallardo, An Empirical Study of Web-Based Inspection
Meetings. ACM/IEEE Int. Symp. on Empirical Software Engineering, Rome,
pp. 244-251 (2003).

\end{enumerate}

\end{review}

\newpage
%%%%%%%%%%%%%%%%%%%%%%%%%%

\begin{review}{}

\noindent{\bf Summary}

The authors examine the expectations, outcomes and challenges of code review as
applied at Microsoft using the CodeFlow tool by surveying and interviewing a
large collection of developers and managers. The study had several components:
analyzing the results of a prior study, observing developers and interviewing
developers, performing a card sort of interview data, examining and sorting
review comments, creating an affinity diagram, and surveying managers and
developers. The authors examine 3 research questions and then provide guidance
on what areas should be further researched.\\

\noindent{\bf Evaluation}

Positives:

\begin{itemize}

\item[+] The authors undertook a large study using multiple techniques and sources of
data, analyzing it using widely applied techniques, and then looking for trends

\item[+] The authors identify differences in the expectations and outcomes of managers
and developers

\item[+] The authors place their work in the context of related work

\item[+] The authors identify the work that should be approached in the future

\item[+] The paper is well written and the approach is explained well

\item[+] The take away message is clear - though developers and managers plan to use
code reviews to find faults, that is not really the main thing that happens.
Reviews are mainly used for code improvement and understanding, with
understanding being a major issue for developers (indicating a need for further
research).

\end{itemize}

Opportunities for improvement:

\begin{itemize}

\item[p.2] Please describe the initial in-field observations

\item[p.3] was there a Hawthorne effect to the ``thought aloud'' portion of the work?

\item[p.4] top left column - please show a sample card?

\item[p.4] top left - is there not possible bias from the first author sorting the
cards to form initial themes?

\item[p.4] middle of left column - how time consuming were the card sorts?

\end{itemize}

Of lesser importance:

There are some typos and minor grammatical issues - a thorough proofread of the
paper will remedy this

p. 9 need text under section VII header


\end{review}

\newpage
%%%%%%%%%%%%%%%%%%%%%%%%%%

\begin{review}{}

\noindent{\bf Summary}

This paper reports the results of an empirical study of what the authors call
``modern code inspections,'' i.e., lightweight, asynchronous inspections carried
out by individuals with the aid of online tools. The authors used a
multi-method approach, employing interviews, a survey, and analyzing review
comments captured by the tool. The results describe the purposes for which
inspections are carried out, what actually happens in the inspection process as
revealed by the comments, and a broad sampling of managers and developers via
the survey. Among the results are that finding defects is (unsurprisingly) the
most frequently mentioned purpose, but unexpectedly, several other purposes are
very nearly as common, such as code improvement, finding alternative solutions,
and learning about the code base. In fact, inspections appear not to be
terribly effective in finding any but the shallowest and most obvious bugs,
unless the code inspectors are already familiar with the code they are reading.
The authors provide recommendations for tool improvements and for future
research that address these overlooked, and potentially more important,
purposes.\\

\noindent{\bf Evaluation}

This is an excellent paper. The topic of inspections, one might think at first,
has been the subject of so much empirical study that there is little left to
learn. This paper shows definitively that there is much more to learn. The
methods were very well suited to the exploratory nature of the work, and the
use of multiple methods, each with different strengths and weaknesses, provides
a very convincing base of evidence for the claims in the paper. For example,
developers pursue a number of purposes in participating in inspections, as the
interviews clearly bring out. The survey of managers provides a contrasting
view, as they share some of these purposes (e.g., finding bugs) but not others
(e.g., finding alternative solutions). The analysis of comments enriches the
results further, showing what developers communicated about, and the sorts of
issues that appeared to absorb their time and energy (e.g., code
understanding), and those that did not (e.g., trying to find deep bugs).

The paper makes great use of triangulation, i.e., seeking additional evidence
to support or refute a tentative interpretation. One of the most interesting
was the observation, from the interviews and comments, that usually only
superficial bugs were found, and another observation that unfamiliar code was
very hard to understand. In the survey of developers, they asked for each of
the inspection outcomes (e.g., finding bugs, learning about the code, improving
the code) how much code understanding was required (rated on a scale from
"none" to "complete"). Interestingly, "finding defects" came out as requiring
the greatest level of understanding, and along with "alternative solutions"
were so far above the others, that they seem qualitatively very different.
Other outcomes one might expect to be similar, such as "code improvement" were
far lower. This finding not only reinforces the earlier results about finding
only superficial bugs and understanding being the most difficult part of the
inspection, it suggests an interesting relationship between the two.

This is an important paper. It provides convincing evidence that inspections
are important, but not for the reasons we thought. In fact, under many
conditions, they are not terribly useful for finding bugs, but they accomplish
many other important purposes. As the authors point out, tools and processes
are optimized around those things that inspections are not very good at, and
they neglect important things like providing assistance for understanding the
code, creating a sense of shared ownership, and transferring knowledge. I think
this paper will change how the field thinks about inspections and what we need
to support them. In fact, it has implications beyond inspections, for example,
the about the potential value of "transparent" environments, such as GitHub,
where all coding is done out in the open, and for pair programming that in
practice can be very close to lightweight inspections, and serves many of the
same purposes.

As an aside, the paper also introduces some interesting analysis techniques.
Open sorting is an interesting variation on the standard qualitative technique
of open coding, and the way they used the affinity diagram - as a technique to
find order in the categories they derived from open coding - is an interesting
way to carry out axial coding. It might be helpful to readers to connect these
techniques with the classical qualitative methods works such as:

Strauss, A., \& Corbin, J. (1990). Basics of Qualitative Research: Grounded
Theory Procedures and Techniques. Newbury Park, CA: Sage Publications, Inc.

A few nits:

\begin{itemize}

\item[p.1] explorative --$>$ exploratory

\item[p.3]: we contacted 100 random candidates - I think you mean they were selected
randomly. The candidates might bristle at being called random . . .

\item[p.9]: ``one common notion . . . '' I don't think this notion is at all common. I
think it is quite out of step with what people believe and the kind of work
that does well in the review process. What is the source of this? I see no
citations. The sentences that follow sound needlessly defensive to me.

\end{itemize}

\end{review}


\end{document}