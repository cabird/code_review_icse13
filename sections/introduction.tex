%!TEX root = ../paper.tex

%%%%%%%%%%%%%%%%%%%%%%
\section{Introduction} \label{sec:introduction}
%%%%%%%%%%%%%%%%%%%%%%

%   \vspace{-0.5em}

Peer code review, a manual inspection of source code by developers other than
the author, is recognized as a valuable tool for reducing software defects and
improving the quality of software projects~\cite{ackerman1984software, ackerman1989software}.
In 1976, Fagan formalized a highly
structured process for code reviewing~\cite{fagan1976design}, based on line-by-line group
reviews, done in extended meetings---\emph{code inspections}. Over the years,
researchers provided evidence on code inspection's benefits, especially in
terms of defect finding, but the cumbersome, time-consuming, and synchronous
nature of this approach hinders its universal adoption in practice~\cite{shull2008inspecting}. %(Shull 2008).

Nowadays many organizations are adopting more lightweight code review
practices to limit the inefficiencies of inspections. In particular, there is a
clear trend toward the usage of tools developed to support code
review~\cite{rigby2012open}. In the context of this paper, we define
\emph{Modern Code Review}, as review that is \begin{inparaenum}[(1)]
\item informal (in contrast to Fagan-style), 
\item tool-based, and that 
\item occurs regularly in practice nowadays, for example at companies such as Microsoft, Google~\cite{kennedy2006online},
Facebook~\cite{tsotsis2011online}, and in other organizations and OSS projects~\cite{gerrit2012online}. \end{inparaenum}

This trend raises questions, such as: What are the
expectations for code review nowadays? What are the actual outcomes of code
review? What challenges do people face in code review?

Answers to these questions can provide insight for both practitioners and
researchers.  Developers and other software project stakeholders can use
empirical evidence about expectations and outcomes to make informed decisions
about when to use code review and how it should fit into their development
process. Researchers can focus their attention on practitioners' challenges to
make code review more effective.

We present an in-depth study of practices in teams that use modern code review,
revealing what practitioners think, do, and achieve when it comes to modern
code review.

Since Microsoft is made up of many different teams working on very diverse
products, it gives the opportunity to study teams performing code review \emph{in
situ} and understand their expectations, the benefits they derive from code
review, the needs they have, and the problems they face.

We set up our study as an exploratory investigation. We started without \emph{a
priori} hypotheses regarding how and why code review should be performed, with
the aim of discovering what developers and managers expect from code review,
how reviews are conducted in practice, and what the actual outcomes and
challenges are. To that end, we \begin{inparaenum}[(1)] 
\item observed 17 industrial developers performing code review with various degrees of experience and seniority across 16 separate product teams with distinct reviewing cultures and policies; 
\item interviewed these developers using a semi-structured interviews; 
\item manually inspected and classified the content of 570 comments in discussions contained within code reviews; and 
\item surveyed 165 managers and 873 programmers.
\end{inparaenum}

Our results show that, although the top motivation driving code reviews is
still finding defects, the practice and the actual outcomes are less about
finding errors than expected: Defect related comments comprise a small
proportion and mainly cover small logical low-level issues. At the same time,
code review additionally provides a wide spectrum of benefits to software
teams, such as knowledge transfer, team awareness, and improved solutions to
problems. Moreover, we found that context and change understanding is the key
of any review. According to the outcomes they want to achieve, developers
employ many mechanisms to fulfill their understanding needs, most of which are
not currently met by any code review tool.

This paper makes the following contributions:
\begin{itemize}
  \item Characterizing the motivations of developers and managers for code review and compare with actual outcomes.
  \item Relating the outcomes to understanding needs and discuss how developers achieve such needs.
\end{itemize}

Based on our findings, we provide recommendations for practitioners and
implications for researchers as well as outline future avenues for research.